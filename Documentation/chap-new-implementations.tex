\chapter{Writing new implementations}

\section{Writing new line implementations}

\subsection{Package}

It is generally a good idea to define a new package for a new
implementation of the concept of a line.  For the remainder of this
section, we use the name \texttt{new-line} for this package.

In general, we advise against the \texttt{:use} of packages other than
the \texttt{common-lisp} package.  For that reason, in the remainder
of this chapter, we use explicit package prefixes to make it clear
what symbols are referred to.

\subsection{Classes}

A class that is a subclass of \texttt{cluffer:line} must be provided.

It is not mandatory to provide any implementation of the \emph{cursor}
abstraction, but if such an abstraction is provided, it is recommended
that the root class of all cursor classes be a subclass of
\texttt{cluffer:cursor}.

\section{Writing new buffer implementations}

