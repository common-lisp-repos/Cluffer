\section{Line edit protocol}
\label{sec-edit-protocol}

\subsection{Protocol classes}

\Defclass {line}

This class is the base class for all lines.  It should not itself be
instantiated.  Instead, \sysname{} contains two different modules
each supplying a different subclass of \texttt{line} that can be
instantiated.

\Defclass {cursor}

This is the base class for all cursors.

\Definitarg {:line}

The class \texttt{cursor} accepts this initarg which is the line to
which the new cursor is to be attached.

\subsection{Operations on lines and cursors}

\Defgeneric {beginning-of-line-p} {cursor}

Return \textit{true} if and only if \textit{cursor} is located at the
beginning of a line.

\Defgeneric {end-of-line-p} {cursor}

Return \textit{true} if and only if \textit{cursor} is located at the
end of a line.

\Defgeneric {beginning-of-line} {cursor}

Position \textit{cursor} at the very beginning of the line.

\Defgeneric {end-of-line} {cursor}

Position \textit{cursor} at the very end of the line.

\Defgeneric {forward-item} {cursor}

Move \textit{cursor} forward one position.  If \emph{cursor} is at the
end of the line, the error condition \texttt{end-of-line} will be
signaled.

\Defgeneric {backward-item} {cursor}

Move \textit{cursor} backward one position.  If \emph{cursor} is at
the beginning of the line, the error condition
\texttt{beginning-of-line} will be signaled.

\Defgeneric {insert-item-at-position} {line item position}

Insert \textit{item} into \textit{line} at \textit{position}.

If \textit{position} is less than zero, a condition of type
\texttt{beginning-of-line} is signaled.  If \textit{position} is
greater than the number of items in \textit{line} (See the definition
of the generic function \texttt{item-count}.), a condition of type
\texttt{end-of-line} is signaled.

After this operation completes, what happens to cursors located at
\textit{position} before the operation depends on the class of the
cursor and of \textit{line}.  The \texttt{standard-line}
implementation provides two kinds of cursors, namely
\emph{left-sticky} cursors and \emph{right-sticky} cursors.  For such
an implementation, after this operation completes, any left-sticky
cursor located at \textit{position} will be located before
\textit{item}, and any right-sticky cursor located \textit{position}
will be located after \textit{item}.

\Defgeneric {delete-item-at-position} {line position}

Delete the item at \textit{position} in \textit{line}.

If \textit{position} is less than zero, a condition of type
\texttt{beginning-of-line} is signaled.  If \textit{position} is
greater than or equal to the number of items in \textit{line} (See the
definition of the generic function \texttt{item-count}.), a condition
of type \texttt{end-of-line} is signaled.

\Defgeneric {insert-item} {cursor item}

Calling this function is equivalent to calling
\texttt{insert-item-at-position} with the line to which
\textit{cursor} is attached, \textit{item}, and the position of
\textit{cursor}.  However, this function might be implemented
differently for reasons of performance.

If \textit{cursor} is not currently attached to a line, a condition
of type \texttt{cursor-detached} is signaled.

\Defgeneric {delete-item} {cursor}

Delete the item immediately after \emph{cursor}.

Calling this function is equivalent to calling
\texttt{delete-item-at-position} with the line to which
\textit{cursor} is attached and the position of \textit{cursor}.
However, this function might be implemented differently for reasons of
performance.

If \textit{cursor} is not currently attached to a line, a condition
of type \texttt{cursor-detached} is signaled.

\Defgeneric {erase-item} {cursor}

Delete the item immediately before \emph{cursor}.

Calling this function is equivalent to calling
\texttt{delete-item-at-position} with the line to which
\textit{cursor} is attached and the position of \textit{cursor} minus
one.  However, this function might be implemented differently for
reasons of performance.

If \textit{cursor} is not currently attached to a line, a condition
of type \texttt{cursor-detached} is signaled.

\Defgeneric {item-after-cursor} {cursor}

Return the item located immediately to the right of \textit{cursor}.
If \emph{cursor} is at the end of the line, the error condition
\texttt{end-of-line} will be signaled.

\Defgeneric {item-before-cursor} {cursor}

Return the item located immediately to the left of \textit{cursor}.
If \emph{cursor} is at the beginning of the line, the error condition
\texttt{beginning-of-line} will be signaled.

\Defgeneric {detach-cursor} {cursor}

The class of \textit{cursor} is changed to \texttt{detached-cursor}
and it is removed from the line it was initially located in. 

If \textit{cursor} is already detached, this operation has no effect.

\Defgeneric {attach-cursor} {cursor line \optional (position 0)}

Attach \textit{cursor} to \textit{line} at \textit{position}.  If
\textit{position} is supplied and it is greater than the number of
items in \textit{line}, the error condition \texttt{end-of-line} is
signaled.  If \textit{cursor} is already attached to a line, the error
condition \texttt{cursor-attached} will be signaled.

\Defgeneric {cursor-position} {cursor}

Return the position of \textit{cursor} as two values: the line number
and the item number within the line. 

\Defgeneric {item-count} {entity}

If \textit{entity} is a line, then return the number of items in that
line.  If \textit{entity} is a cursor, return the number of items in
the line in which \textit{cursor} is located.

Note: the argument \textit{entity} can also be a buffer, in which case
the total number of items is returned.
\seesec{sec-buffer-edit-protocol}

\Defgeneric {line} {cursor}

Return the line in which \textit{cursor} is located. 
