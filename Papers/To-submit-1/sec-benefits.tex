\section{Benefits of our technique}
 
There are several advantages to our technique compared to other
existing solutions.

First, most techniques expose a more concrete representation of the
buffer to client code, such as a doubly linked list of lines.  Our
technique is defined in terms of an abstract \clos{} protocol that can
have several potential implementations.

Furthermore, our \emph{update protocol} based on time stamps provides
an elegant solution to the problem of updating multiple views at
different times and with different frequencies.  As opposed to the
technique of using \emph{observers} preferred in the object-oriented
literature, time stamps require no communication from the model to the
views as a result of modifications; indeed, such communication would
be undesirable because of the high frequency of modifications to the
model compared to the frequency of view updates.  The standard buffer
implementation provided by our library provides an efficient
implementation of the update protocol.

Our technique can be \emph{customized} by the fact that the buffer
editing protocol and the line editing protocol are independent.
Client code with specific needs can therefore replace the
implementation of one or the other or both according to its
requirements.  Thanks to the existence of the \clos{} protocol, such
customization can be done gradually, starting with the supplied
implementations and replacing them as requirements change.

The standard line implementation supplied makes it possible to obtain
reasonable performance for aggregate editing operations even when
these operations are implemented as iterative calls to elementary
editing operations.  This quality makes it possible for client code to
be simpler, for obvious benefits.

Finally, our technique is not specific to the abstractions of any
particular existing editor, making our library useful in a variety of
potential clients.
