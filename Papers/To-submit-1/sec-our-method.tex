\section{Our technique}
\label{sec-our-technique}

\subsection{Protocols}

Recall from \refSec{sec-introduction} the existence of two nested
control loops, the \emph{inner} control loops in which each iteration
is executing a single edit operation, and the \emph{outer} control
loop for the purpose of updating views.

The inner control loop is catered to by two different protocols; one
containing operations on individual \emph{lines} of items and one
containing operations at the \emph{buffer} level, concerning mainly
the creation and deletion of lines.  While we supply reasonable
implementations of both these protocols, we also allow for
sophisticated clients to substitute specific implementations of each
one.

The outer control loop is catered to by the \emph{update protocol}.
This protocol is based on the concept of \emph{time stamps}.  In order
to request an update, client code supplies the time stamp of the
previous similar request in addition to four different functions
(\texttt{sync}, \texttt{skip}, \texttt{modify}, and \texttt{create}).
These functions can be thought of as representing editing operations
on the lines of the buffer.  Our protocol implementation calls these
functions in an order that will update the buffer contents from its
previous to its current state.  The implementations of these functions
are supplied by client code according to its own representation of the
buffer contents.

\refFig{fig-external-protocols} illustrates the relationship between
these protocols.

\begin{figure}
\begin{center}
\inputfig{fig-external-protocols.pdf_t}
\end{center}
\caption{\label{fig-external-protocols}
External protocols.}
\end{figure}

The protocols illustrated in \refFig{fig-external-protocols} are
related to one another by the \emph{protocol classes} that they
operate on.  The buffer-edit protocol operates on instances of the
protocol class named \texttt{buffer}.  The line-edit protocol operates
on instances of the two protocol classes \texttt{line} and
\texttt{cursor}.  These protocols are tied together using an internal
protocol class named \texttt{dock}.  \refFig{fig-participation}
illustrates the participation of these protocol classes in the
different protocols, omitting the update protocol.

\begin{figure}
\begin{center}
\inputfig{fig-participation.pdf_t}
\end{center}
\caption{\label{fig-participation}
Participation of classes in protocols.}
\end{figure}

The internal protocol contains generic functions for which methods
must be created that specialize to different \emph{implementations} of
the buffer-edit and the line-edit protocols.  Client code using the
library is not concerned with the existence of the internal protocol.

\subsection{Supplied implementations}

For the \emph{line protocol}, we supply two different implementations,
the \emph{standard line} implementation and the \emph{simple line}
implementation.  Similarly, for the \emph{buffer protocol}, we supply
two different implementations, the \emph{standard buffer} implementation
and the \emph{simple buffer} implementation.

\subsubsection{Standard line implementation}

The standard line implementation is the one that a typical application
would always use, unless an application-specific line implementation
is desired.

To appreciate the design of the standard line, we need to distinguish
between two different \emph{categories} of operations on a line.  We
call these categories \emph{editing operations} and \emph{contents
  queries}, respectively.  An editing operation is one in which the
contents of the line is modified in some way, and is the result of the
interaction of a user typing text, inserting or removing a
\emph{region} of text, or executing a \emph{keyboard macro} that
results in one or more editing operations.  A contents query happens
as a result of an \emph{event loop} or a \emph{command loop} updating
one or more \emph{views} of the contents.

A crucial observation related to these categories is that contents
queries are the result of \emph{events}; typically the user typing
text or executing commands.  The frequency of such events is fairly
low, giving us ample time to satisfy such a query.  Editing
operations, on the other hand, can be arbitrarily more frequent,
simply because a single keystroke on the part of the user can trigger
a very large number of editing operations.%
\footnote{It is of course possible to supply \emph{aggregate}
  operations that alleviate the problem of frequent editing
  operations.  In particular, it is possible to supply operations that
  insert a \emph{sequence} of items, and that delete a \emph{region}
  of items.  However, such operations complicate the implementations
  of the protocol.  Worse, there are still cases where many simple
  editing operations need to be executed, in particular as a result of
  executing keyboard macros.}

This implementation supplies two different representations of the line
that we call \emph{open} and \emph{closed} respectively.  A line is
\emph{open} if the last operation on it was an editing operation.  It
is \emph{closed} if the last operation was a contents query in the
form of a call to the generic function \texttt{items}.  Accordingly, a
line is changed from being open to being closed whenever there is a
contents query, and from closed to open when there is call to an
editing operation.

A closed line is represented as a \commonlisp{} simple vector.  An
open line is represented as a gap buffer. \seesec{sec-previous-work}
The protocol specifically does not allow for the caller of a contents
query to modify the vector returned by the query.  This restriction
allows us to return the same vector each time there is a contents
query without any intervening editing operation, thus making it
efficient for views to query closed lines repeatedly.  Similarly,
repeated editing operations maintain the line open, making such a
sequence of operations efficient as well.

Clearly, the typical use case when a user issues keystrokes, each one
resulting in a simple editing operation such as inserting or deleting
an item, followed by an update of one or more views of the buffer
contents is not terribly efficient.  The reason is that this use case
results in a line being alternately opened (as a result of the editing
operation) and closed (as a result of the view update) for each
keystroke.  However, this use case does not have to be very efficient,
again because it all happens at the frequency of the event loop.  The
use case for which the standard line design was optimized is the one
where a single keystroke results in several simple editing operations,
which is exactly when performance is crucial.

\subsubsection{Simple line implementation}

We supply a second implementation for the line editing protocol,
called the \emph{simple line}.  The main purpose of this
implementation is to serve as a reference for \emph{random tests}.
The idea here is that the implementation of the simple line is
trivial, so that the correctness of the implementation is mostly
obvious from inspecting the code, and in any case, it is unlikely that
a defect in the simple line and another defect in the standard
line will result in the same external behavior on a large body of
randomly-generated operations.

In addition to serving as a reference implementation for testing the
standard line, the implementation of the simple line can also serve as
a reference for programmers who would like to create their own
implementation of the line editing protocol.

The simple line implementation provides a single line abstraction,
implemented as a \commonlisp{} simple vector.  Each editing operation
is implemented as reallocation of a new vector followed by calls to
\texttt{replace} to copy items from the original line contents to the
one resulting from the editing operation.  Clearly, this technique is
very inefficient.  For that reason, it is not recommended to use the
simple implementation in client code.

\subsubsection{Standard buffer implementation}

The main performance challenge for the buffer implementation is to
obtain acceptable performance in the presence of multiple views (into
a single buffer) that are far apart, and that both issue editing
operations in each interaction.  The typical scenario would be a user
having two views, one close to the beginning of the buffer and one
close to the end of the buffer, while executing a keyboard macro that
deletes from one of the views and inserts into the other.

This time, the performance challenge has to do with the \emph{update
  protocol} rather than with the edit protocols.  A naive buffer
implementation would have to iterate over all the lines each time the
update protocol is invoked.

To obtain reasonable performance in the presence of multiple views,
the standard buffer implementation uses a \emph{splay tree}
\cite{Sleator:1985:SBS:3828.3835} with a node for each line in the
buffer.  A splay tree is a \emph{self adjusting} binary tree, in that
nodes that are frequently used migrate close to the root of the tree.
Although the typical use of splay trees and other tree types is to
serve as implementations of \emph{dictionaries}, an often overlooked
fact is that all trees can be used to implement \emph{editable
  sequences}, which is how we use the splay tree here.

In addition to containing a reference to the associated line, each
node in the splay tree contains time stamps corresponding to when the
line was created and last modified.  In addition, each node also
contains summary information for the entire subtree rooted at this
node.  This summary information is what allows us to skip entire
subtrees when a view requests update information and no node in the
subtree has been modified since the last update request.

Finally, each node contains both a line count and an item count for
the entire subtree, so that the offset of a particular line or a
particular item can be computed efficiently, at least for nodes that
are close to the root of the tree.

\subsubsection{Simple buffer implementation}

As with the implementations of the line-edit protocol, we supply a
second implementation for the buffer editing protocol as well, called
the \emph{simple buffer}.  Again, the main purpose of this
implementation is to serve as a reference for \emph{random tests}.  As
with the simple line implementation, the implementation of the simple
buffer is trivial, so that the correctness of the implementation is
mostly obvious from inspecting the code.

The simple buffer implementation represents the buffer as a
\commonlisp{} vector of nodes, where each node contains a line and
time stamps indicating when a line was created and last modified.
