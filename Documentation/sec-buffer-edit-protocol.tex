\section{Buffer edit protocol}
\label{sec-buffer-edit-protocol}

\Defclass {buffer}

This is the class to instantiate in order to create a buffer.

\Defgeneric {beginning-of-buffer-p} {cursor}

Return \textit{true} if and only if \textit{cursor} is located at the
beginning of a buffer.

\Defgeneric {end-of-buffer-p} {cursor}

Return \textit{true} if and only if \textit{cursor} is located at the
end of a buffer.

\Defgeneric {beginning-of-buffer} {cursor}

Position \textit{cursor} at the very beginning of the buffer.

\Defgeneric {end-of-buffer} {cursor}

Position \textit{cursor} at the very end of the buffer.

\Defgeneric {split-line} {cursor}

Split the line in which \textit{cursor} is located into two lines, the
first cone containing the text preceding \textit{cursor} and the
second one containing the text following \textit{cursor}.  After this
operation, any left-sticky cursor located at the same position as
\textit{cursor} will be located at the end of the first line, and any
right-sticky cursor located at the same position as \textit{cursor}
will be located at the beginning of the second line.

\Defgeneric {join-line} {cursor}

Join the line in which \textit{cursor} is located and the following
line.  If \textit{cursor} is located at the last line of the buffer,
the error condition \texttt{end-of-buffer} will be signaled.

\Defgeneric {line-count} {cursor}

Return the number of lines in the buffer in which \textit{cursor} is
located.

\Defgeneric {line-number} {entity}

The argument \textit{entity} may be a cursor or a line.

If \textit{entity} is a cursor and that cursor is not attached to any
line, a condition of type \texttt{cursor-detached} is signaled.

If \textit{entity} is a cursor and that cursor is attached to a line,
then the generic function \texttt{line} \seesec{sec-edit-protocol} is
called with \textit{entity} as an argument and the return value is
used as argument in a recursive call to \texttt{line-number}.

If \textit{entity} is a line and that line is not attached to a
buffer, then \texttt{nil} is returned.

If \textit{entity} is a line and that line is attached to a buffer,
then the line number of \textit{line} in that buffer is returned.  The
first line of the buffer has the number $0$.

\Defgeneric {find-line} {buffer line-number}

Return the line in the buffer with the given \textit{line-number}.  If
\textit{line-number} is less than $0$ then the error
\texttt{beginning-of-buffer} is signaled.  If \textit{line-number} is
greater than or equal to the number of lines in the buffer, then the
error \texttt{end-of-buffer} is signaled.

\Defgeneric {buffer} {object}

Return the buffer in which \textit{object} is located, where
\textit{object} is a \emph{cursor} or a \emph{line}.

Notice that the edit protocol does not contain any
\texttt{delete-line} operation.  This design decision was made on
purpose.  By only providing \texttt{join-line}, we guarantee that
removing a line leaves a \emph{trace} in the buffer in the form of a
modification operation on the first of the two lines that were
joined.  This features is essential in order for the \emph{update
  protocol} to work correctly.
