\section{Previous work}
\label{sec-previous-work}

\subsection{GNU Emacs}

GNU Emacs uses a so-called \emph{gap buffer} for the entire buffer of
text \cite{GNUEmacsLispReferenceManual} \cite{CraftOfTextEditiing}.  A
gap buffer can be thought of as a vector holding the items of the
buffer (typically characters) but with some additional free space.  In
a typical gap-buffer implementation, a possibly-empty \emph{prefix} of
the buffer contents is stored at the beginning of the vector, and a
possibly-empty \emph{suffix} of the contents is stored at the end of
the vector, leaving a possibly-empty \emph{gap} between the prefix and
the suffix.

Buffer items are moved from the end of the prefix to the beginning of
the suffix, and vice-versa, in order to position the gap where an item
is about to be inserted or deleted.  The typical use case for text
editing has a very high probability that two subsequent editing
operations will be \emph{close} to each other (in terms of the number
of items between the two.  Therefore, this data structure is very
efficient for editing operations in most cases.

Clearly, in the worst case, all buffer items must be moved for every
editing operation.  This case happens when editing operations
alternate between the beginning of the buffer and the end of the
buffer.  Even so, moving all the items even in a very large buffer
does not represent a serious performance problem.  Furthermore, the
pathological case can be largely avoided by considering the vector
holding the items as being \emph{circular} as Flexichain
\cite{flexichain} does.

Perhaps the main disadvantage of representing the entire buffer as a
single gap buffer is that it is difficult to associate additional
information with specific points in the buffer.  One might, for
instance, want to associate some state of an \emph{incremental parser}
that keeps track of the buffer contents in a more structured form.

Another difficulty with the gap-buffer representation has to do with
updating possibly multiple \emph{views}.  As we discussed in
\refSec{sec-introduction}, views are updated at the frequency of the
event loop, whereas the manipulation of \emph{regions} of items and
especially the use of \emph{keyboard macros} may make the frequency of
editing operations orders of magnitude higher.  When GNU Emacs was
designed, the main issue with updating a view was to minimize the
number of bytes that had to be sent to a CRT terminal.  To accomplish
this optimization, the \emph{redisplay} function compared the previous
view to the next one, and attempted to issue terminal-specific editing
operations to turn the screen contents into the updated version.  Of
course, most of the time, the task consisted of positioning the cursor
and inserting a single character.

Today, there is no need to minimize the number of editing operations
on a terminal; it is perfectly feasible to redraw the entire view for
each iteration of the event loop.  However, today we have many more
requirements on a text editor.  In the most advanced cases, we would
like for an \emph{incremental parser} in the view to keep a structured
version of the buffer contents, for various purposes, such as syntax
highlighting, language-specific completion and parsing, etc.  An
incremental parser may require considerable computing power.  It is
therefore of utmost importance that as little work as possible is done
each time around the event loop.  Representing the entire editor
buffer as a gap buffer does not lend itself to such advanced
incremental processing.

\subsection{Multics Emacs}


Multics Emacs%
\footnote{The description in this section is a summary of the
  information found here: http://www.multicians.org/mepap.html}
was the first Emacs implementation written in \lisp{}, in this case
\multics{} \maclisp{}.  In therefore pre-dates GNU Emacs.

Multics Emacs used a doubly-linked list of lines and each line was a
sequence of characters.  Special instructions were added to the GE
645...

\subsection{Climacs}

Like GNU Emacs, Climacs uses a gap buffer for the entire buffer.  It
avoids the bad case by using a circular buffer.  In fact, it uses
Flexichain \cite{flexichain}.

\subsection{Others}

Hemlock, Goatee, etc.

