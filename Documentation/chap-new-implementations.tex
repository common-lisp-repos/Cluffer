\chapter{Writing new implementations}

\section{Introduction}

In general, we advise against the \texttt{:use} of packages other than
the \texttt{common-lisp} package.  For that reason, in the remainder
of this chapter, we use explicit package prefixes to make it clear
what symbols are referred to.

\section{Writing new line implementations}

\subsection{Package}

It is generally a good idea to define a new package for a new
implementation of the concept of a line.  For the remainder of this
section, we use the name \texttt{new-line} for this package.

\subsection{Classes}

A class that is a subclass of \texttt{cluffer:line} must be provided.
In the remainder of this section we refer to this class as
\texttt{new-line:line}.

It is not mandatory to provide any implementation of the \emph{cursor}
abstraction, but if such an abstraction is provided, it is recommended
that the root class of all cursor classes be a subclass of
\texttt{cluffer:cursor}.  In the remainder of this section we refer to
this class as \texttt{new-line:cursor}.

Both the \emph{standard line}
\seesec{sec-implementation-standard-line} and the \emph{simple line}
\seesec{sec-implementation-simple-line} provide two kinds of cursor
abstractions, namely \emph{left-sticky} cursors and
\emph{right-sticky} cursors.  If the new implementation provides
different kinds of cursors, the different behavior can be accomplished
either by subclassing as is done in both provided implementations, or
it can be accomplished by some other means such as storing additional
information in slots of the root class.

\subsection{Methods}

\Defmethod {item-count} {(line \texttt{new-line:line})}

This method should return the number of items in \textit{line}.  If
the new implementation defines subclasses of \texttt{new-line:line}
for which the item count is computed differently, as is the case for
the standard line \seesec{sec-implementation-standard-line} then this
method must be replaced by a method for each subclass with a specific
way of computing the item count.

\section{Writing new buffer implementations}

