\section{Previous work}

\subsection{GNU Emacs}

GNU Emacs uses a so-called \emph{gap buffer} for the entire buffer of
text \cite{GNUEmacsLispReferenceManual} \cite{CraftOfTextEditiing}.  A
gap buffer can be thought of as a vector holding the items of the
buffer (typically characters) but with some additional free space.  In
a typical gap-buffer implementation, a possibly-empty \emph{prefix} of
the buffer contents is stored at the beginning of the vector, and a
possibly-empty \emph{suffix} of the contents is stored at the end of
the vector, leaving a possibly-empty \emph{gap} between the prefix and
the suffix.

Buffer items are moved from the end of the prefix to the beginning of
the suffix, and vice-versa, in order to position the gap where an item
is about to be inserted or deleted.  The typical use case for text
editing has a very high probability that two subsequent editing
operations will be \emph{close} to each other (in terms of the number
of items between the two.  Therefore, this data structure is very
efficient for editing operations in most cases.

Clearly, in the worst case, all buffer items must be moved for every
editing operation.  This case happens when editing operations
alternate between the beginning of the buffer and the end of the
buffer.  Even so, moving all the items even in a very large buffer
does not represent a serious performance problem.

\subsection{Multics Emacs}


Multics Emacs%
\footnote{The description in this section is a summary of the
  information found here: http://www.multicians.org/mepap.html}
used doubly-linked list of lines and each line was a
sequence of characters.  Special instructions were added to the GE
645...

\subsection{Climacs}

Like GNU Emacs, Climacs uses a gap buffer for the entire buffer.  It
avoids the bad case by using a circular buffer.  If fact, it uses
Flexichain \cite{flexichain}.

\subsection{Others}

Hemlock, Goatee, etc.

