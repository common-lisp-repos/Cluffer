\documentclass{beamer}
\usepackage[utf8]{inputenc}
\beamertemplateshadingbackground{red!10}{blue!10}
%\usepackage{fancybox}
\usepackage{epsfig}
\usepackage{verbatim}
\usepackage{url}
%\usepackage{graphics}
%\usepackage{xcolor}
\usepackage{fancybox}
\usepackage{moreverb}
%\usepackage[all]{xy}
\usepackage{listings}
\usepackage{filecontents}
\usepackage{graphicx}

\lstset{
  language=Lisp,
  basicstyle=\scriptsize\ttfamily,
  keywordstyle={},
  commentstyle={},
  stringstyle={}}

\def\inputfig#1{\input #1}
\def\inputeps#1{\includegraphics{#1}}
\def\inputtex#1{\input #1}

\inputtex{logos.tex}

%\definecolor{ORANGE}{named}{Orange}

\definecolor{GREEN}{rgb}{0,0.8,0}
\definecolor{YELLOW}{rgb}{1,1,0}
\definecolor{ORANGE}{rgb}{1,0.647,0}
\definecolor{PURPLE}{rgb}{0.627,0.126,0.941}
\definecolor{PURPLE}{named}{purple}
\definecolor{PINK}{rgb}{1,0.412,0.706}
\definecolor{WHEAT}{rgb}{1,0.8,0.6}
\definecolor{BLUE}{rgb}{0,0,1}
\definecolor{GRAY}{named}{gray}
\definecolor{CYAN}{named}{cyan}

\newcommand{\orchid}[1]{\textcolor{Orchid}{#1}}
\newcommand{\defun}[1]{\orchid{#1}}

\newcommand{\BROWN}[1]{\textcolor{BROWN}{#1}}
\newcommand{\RED}[1]{\textcolor{red}{#1}}
\newcommand{\YELLOW}[1]{\textcolor{YELLOW}{#1}}
\newcommand{\PINK}[1]{\textcolor{PINK}{#1}}
\newcommand{\WHEAT}[1]{\textcolor{wheat}{#1}}
\newcommand{\GREEN}[1]{\textcolor{GREEN}{#1}}
\newcommand{\PURPLE}[1]{\textcolor{PURPLE}{#1}}
\newcommand{\BLACK}[1]{\textcolor{black}{#1}}
\newcommand{\WHITE}[1]{\textcolor{WHITE}{#1}}
\newcommand{\MAGENTA}[1]{\textcolor{MAGENTA}{#1}}
\newcommand{\ORANGE}[1]{\textcolor{ORANGE}{#1}}
\newcommand{\BLUE}[1]{\textcolor{BLUE}{#1}}
\newcommand{\GRAY}[1]{\textcolor{gray}{#1}}
\newcommand{\CYAN}[1]{\textcolor{cyan }{#1}}

\newcommand{\reference}[2]{\textcolor{PINK}{[#1~#2]}}
%\newcommand{\vect}[1]{\stackrel{\rightarrow}{#1}}

% Use some nice templates
\beamertemplatetransparentcovereddynamic

\newcommand{\A}{{\mathbb A}}
\newcommand{\degr}{\mathrm{deg}}

\title{A CLOS Protocol for Editor Buffers}

\author{Robert Strandh}
\institute{
LaBRI, University of Bordeaux
}
\date{May, 2016}

%\inputtex{macros.tex}


\begin{document}
\frame{
\resizebox{3cm}{!}{\includegraphics{Logobx.pdf}}
\hfill
\resizebox{1.5cm}{!}{\includegraphics{labri-logo.pdf}}
\titlepage
\vfill
\small{European Lisp Symposium, Cracow, Poland \hfill ELS2016}
}

\setbeamertemplate{footline}{
\vspace{-1em}
\hspace*{1ex}{~} \GRAY{\insertframenumber/\inserttotalframenumber}
}

\frame{
\frametitle{An editor buffer as an \emph{editable sequence}}

\begin{itemize}
\item An editor buffer can be considered to be an \emph{editable
  sequence} of items.
\item Solutions with optimal asymptotically worst-case complexity are
  too slow and/or have too much space overhead.
\item We must take advantage of typical use cases to do better.
\item Two main representations:
  \begin{enumerate}
  \item Gap buffer.
  \item Line oriented.
  \end{enumerate}
\end{itemize}
}

\frame{
  \frametitle{Gap buffer}

  \begin{itemize}
  \item The contents of the buffer is stored in a \emph{vector}.
  \item Three parts:
    \begin{enumerate}
    \item A possibly empty \emph{prefix} of items.
    \item A possibly empty \emph{gap}.
    \item A possibly empty \emph{suffix} of items.
    \end{enumerate}
  \item Items are moved so that modifications are either at the
    beginning or at the end of the gap.
  \end{itemize}

}

\frame{
  \frametitle{Line oriented}

  \begin{itemize}
  \item Representation as a two-level editable sequence.
  \item The outer level is a sequence of lines.
  \item At the inner level, each line is a sequence of items.
  \item The outer level can be a gap buffer, but can also use
    a traditional representation such as a tree.
  \end{itemize}
}

\frame{
  \frametitle{Existing editors}

  \begin{itemize}
  \item GNU Emacs and (first) Climacs use a gap buffer to represent
    the entire editor buffer.
  \item Many others use a doubly linked list of lines, such as Multics
    Emacs, Hemlock, Goatee.
  \item VIM uses a tree of file-backed blocks.
  \end{itemize}
}

\frame{
  \frametitle{Two loops}

  \begin{itemize}
  \item The slow loop.  It is typically triggered by a keystroke, but
    can also be triggered by other events such as mouse events or when
    a window is re-sized.
  \item The fast loop.  Each iteration is an operation to
    insert/delete an item, or to move a cursor.
  \end{itemize}
}

\frame{
  \frametitle{The slow loop}

  \begin{itemize}
  \item Sufficiently slow that the entire view can be redrawn for each
    iteration.
  \item The view must determine what has changed in the buffer since 
    it was last redrawn. 
  \end{itemize}
}

\frame{
  \frametitle{The fast loop}

  \begin{itemize}
  \item The fast loop is faster because a single keystroke can trigger
    several operations, for example when a region is inserted/deleted,
    or when a keyboard macro is executed.
  \item Some fast cases could be avoided by the use of special
    operations on regions.
  \item With keyboard macros, multiple operations can not easily be
    avoided.
  \item Besides, for reasons of maintainability, it is best to avoid
    special cases.
  \end{itemize}
}

\frame{
  \frametitle{Two protocols}

  \begin{itemize}
  \item The \emph{edit} protocol corresponding to the fast loop.
  \item The \emph{update} protocol corresponding to the slow loop.
  \end{itemize}
}

\frame{
  \frametitle{Acknowledgments}

We would like to thank Daniel Kochmański, Bart Botta, and Matthew Alan
Martin for providing valuable feedback on early versions of this
paper.
}

\frame{
\frametitle{Thank you}

Questions?
}

%% \frame{\tableofcontents}
%% \bibliography{references}
%% \bibliographystyle{alpha}

\end{document}
