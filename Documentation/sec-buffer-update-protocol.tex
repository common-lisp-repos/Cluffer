\section{Buffer update protocol}
\label{sec-update-protocol}

The purpose of the buffer update protocol is to allow for a number of
edit operations to the buffer without updating the view of the buffer.
This functionality is important because a single command may result in
an arbitrary number of edit operations to the buffer, and we typically
want the view to be updated only once, when all those edit operations
have been executed.

At the center of the update protocol is the concept of a \emph{time
  stamp}.  The nature of this time stamp is not specified, other than
the fact that its value is incremented for each operation that alters
the contents of the buffer.  The only operation allowed by client code
on a time stamp is to store it and pass it as an argument to the
protocol function \texttt{update}.

\Defgeneric {current-time} {buffer}

Return the current time of the buffer as a time stamp.

\Defgeneric {items} {line \key start end}

Return the items of \textit{line} as a vector.  The keyword parameters
\textit{start} and \textit{end} have the same interpretation as for
\commonlisp{} sequence functions.

\Defgeneric {update} {buffer time sync skip modify create}

The \textit{buffer} parameter is a buffer that might have been
modified since the last update operation.  The \textit{time} parameter
is the last time the update operation was called, so that the
\texttt{update} function will report modifications since that time. 

The parameters \textit{sync}, \textit{skip}, \textit{modify}, and
\textit{create}, are functions that are called by the \texttt{update}
function.  They are to be considered as \emph{edit operations} on some
representation of the buffer as it was after the previous call to
\texttt{update}.  The operations have the following meaning:

\begin{itemize}
\item \textit{create} indicates a line that has been created.  The
  function is called with that line as an argument.
\item \textit{modify} indicates a line that has been modified.  The
  function is called with that line as an argument.
\item \textit{sync} indicates the first unmodified line after a
  sequence of new or modified lines.  Accordingly, this function is
  called once, following one or more calls to \textit{create} or
  \textit{modify}.  This function is called with a single argument:
  the unmodified line.
\item \textit{skip} indicates that a number of lines have not been
  subject to any modifications since the last update.  The function
  takes a single argument, the number of lines to skip.  This function
  is called \emph{first} to indicate that a \emph{prefix} of the
  buffer is unmodified, or after a \emph{sync} operation to indicate
  that that many lines following the one given as argument to the
  \textit{sync} operation are unmodified.  This operation is also
  called when there are unmodified lines at the end of the buffer so
  that the total line count of the buffer corresponds to the total
  number of lines mentioned in the sequence of operations.
\end{itemize}

