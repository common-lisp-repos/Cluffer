\chapter{Buffer edit protocol}
\label{chap-edit-protocol}

\Defgeneric {beginning-of-buffer-p} {cursor}

Return \textit{true} if and only if \textit{cursor} is located at the
beginning of a buffer.

\Defgeneric {end-of-buffer-p} {cursor}

Return \textit{true} if and only if \textit{cursor} is located at the
end of a buffer.

\Defgeneric {beginning-of-line-p} {cursor}

Return \textit{true} if and only if \textit{cursor} is located at the
beginning of a line.

\Defgeneric {end-of-line-p} {cursor}

Return \textit{true} if and only if \textit{cursor} is located at the
end of a line.

\Defgeneric {beginning-of-buffer} {cursor}

Position \textit{cursor} at the very beginning of the buffer.

\Defgeneric {end-of-buffer} {cursor}

Position \textit{cursor} at the very end of the buffer.

\Defgeneric {beginning-of-line} {cursor}

Position \textit{cursor} at the very beginning of the line.

\Defgeneric {end-of-line} {cursor}

Position \textit{cursor} at the very end of the line.

\Defgeneric {forward-item} {cursor}

Move \textit{cursor} forward one position.  If \emph{cursor} is at the
end of the line, the error condition \texttt{end-of-line} will be
signaled.

\Defgeneric {backward-item} {cursor}

Move \textit{cursor} backward one position.  If \emph{cursor} is at
the beginning of the line, the error condition
\texttt{beginning-of-line} will be signaled.

\Defgeneric {insert-item} {cursor item}

Insert an item at the location of \textit{cursor}.  After this
operation, any left-sticky cursor located at the same position as
\textit{cursor} will be located before \textit{item}, and any
right-sticky cursor located at the same position as \textit{cursor}
will be located after \textit{item}.

\Defgeneric {delete-item} {cursor}

Delete the item immediately to the right of \emph{cursor}.  If
\emph{cursor} is at the end of the line, the error condition
\texttt{end-of-line} will be signaled.

\Defgeneric {erase-item} {cursor}

Delete the item immediately to the left of \emph{cursor}.  If
\emph{cursor} is at the beginning of the line, the error condition
\texttt{beginning-of-line} will be signaled.

\Defgeneric {item-after-cursor} {cursor}

Return the item located immediately to the right of \textit{cursor}.
If \emph{cursor} is at the end of the line, the error condition
\texttt{end-of-line} will be signaled.

\Defgeneric {item-before-cursor} {cursor}

Return the item located immediately to the left of \textit{cursor}.
If \emph{cursor} is at the beginning of the line, the error condition
\texttt{beginning-of-line} will be signaled.

\Defgeneric {split-line} {cursor}

Split the line in which \textit{cursor} is located into two lines, the
first cone containing the text preceding \textit{cursor} and the
second one containing the text following \textit{cursor}.  After this
operation, any left-sticky cursor located at the same position as
\textit{cursor} will be located at the end of the first line, and any
right-sticky cursor located at the same position as \textit{cursor}
will be located at the beginning of the second line.

\Defgeneric {join-line} {cursor}

Join the line in which \textit{cursor} is located and the following
line.  If \textit{cursor} is located at the last line of the buffer,
the error condition \texttt{end-of-buffer} will be signaled.

\Defgeneric {detach-cursor} {cursor}

The class of \textit{cursor} is changed to \texttt{detached-cursor}
and it is removed from the line it was initially located in. 

If \textit{cursor} is already detached, this operation has no effect.

\Defgeneric {attach-cursor} {cursor line \optional (position 0)}

Attach \textit{cursor} to \textit{line} at \textit{position}.  If
\textit{position} is supplied and it is greater than the number of
items in \textit{line}, the error condition \texttt{end-of-line} is
signaled.  If \textit{cursor} is already attached to a line, the error
condition \texttt{cursor-attached} will be signaled.

\Defgeneric {cursor-position} {cursor}

Return the position of \textit{cursor} as two values: the line number
and the item number within the line. 

\Defgeneric {line-count} {cursor}

Return the number of lines in the buffer in which \textit{cursor} is
located.

\Defgeneric {item-count} {cursor}

Return the number of items in the line in which \textit{cursor} is
located.

\Defgeneric {line} {cursor}

Return the line in which \textit{cursor} is located. 

\Defgeneric {line-number} {line}

Return the line number of \textit{line}.  The first line of the buffer
has the number $0$. 

\Defgeneric {find-line} {buffer line-number}

Return the line in the buffer with the given \textit{line-number}.  If
\textit{line-number} is less than $0$ then the error
\texttt{beginning-of-buffer} is signaled.  If \textit{line-number} is
greater than or equal to the number of lines in the buffer, then the
error \texttt{end-of-buffer} is signaled.

\Defgeneric {buffer} {object}

Return the buffer in which \textit{object} is located, where
\textit{object} is a \emph{cursor} or a \emph{line}.

Notice that the edit protocol does not contain any
\texttt{delete-line} operation.  This design decision was made on
purpose.  By only providing \texttt{join-line}, we guarantee that
removing a line leaves a \emph{trace} in the buffer in the form of a
modification operation on the first of the two lines that were
joined.  This features is essential in order for the \emph{update
  protocol} to work correctly.
